\documentclass{article}

% if you need to pass options to natbib, use, e.g.:
%     \PassOptionsToPackage{numbers, compress}{natbib}
% before loading neurips_2020

% ready for submission
% \usepackage{neurips_2020}

% to compile a preprint version, e.g., for submission to arXiv, add add the
% [preprint] option:
\usepackage[preprint]{neurips_2020}

% to compile a camera-ready version, add the [final] option, e.g.:
%     \usepackage[final]{neurips_2020}

% to avoid loading the natbib package, add option nonatbib:
%    \usepackage[nonatbib]{neurips_2020}
\usepackage{ctex}
\usepackage{xurl}
\usepackage{minted}
\usepackage{graphicx}
\usepackage{pdfcomment}		% generate PDF index.
\usepackage[utf8]{inputenc} % allow utf-8 input
\usepackage[T1]{fontenc}    % use 8-bit T1 fonts
\usepackage{hyperref}       % hyperlinks
\usepackage{url}            % simple URL typesetting
\usepackage{booktabs}       % professional-quality tables
\usepackage{amsfonts}       % blackboard math symbols
\usepackage{nicefrac}       % compact symbols for 1/2, etc.
\usepackage{microtype}      % microtypography
\usepackage{float}
\bibliographystyle{unsrt}

\newtheorem{theorem}{Theorem}
\newtheorem{proposition}{Proposition}
\newtheorem{lemma}{Lemma}
\newtheorem{corollary}{Corollary}
\newtheorem{remark}{Remark}
\newtheorem{assumption}{Assumption}
\newtheorem{definition}{Definition}

\title{基于 Rust 的深度学习框架设计:Raddar}

\author{% Reviese your personal information here
葛煦旸 \\ % Your name 
计算机科学技术学院 \\ % CS 
ID: 20307140003 \\
\texttt{20307140003@fudan.edu.cn} \\
\AND
许冬 \\ % Your name 
计算机科学技术学院 \\ % CS 
ID: 20307140036 \\
\texttt{dongxu20@fudan.edu.cn} \\
\AND
舒文韬 \\ % Your name 
计算机科学技术学院 \\ % CS 
ID: 20307140057 \\
\texttt{20307140057@fudan.edu.cn} \\
}

\begin{document}

\maketitle

\begin{abstract}
	
\end{abstract}

\section{背景介绍}
\subsection{生态研究}
\subsection{相关项目}
\subsubsection{tch-rs}
\subsubsection{RustCUDA}

\section{设计思路}
\subsection{易用性}
\subsection{高效性}

\section{实现思路}
\subsection{张量}
\subsection{数据处理}
\subsubsection{数据集}
\subsubsection{数据加载器}
\subsection{模型}
\subsection{优化器}

\section{结论和前景}


\section{贡献}
本项目由三人合作完成,\textbf{排名不分先后}。各合作者的主要贡献部分见表 \ref{tab:contrib}。

\begin{table}[H]
	\centering
	\caption{贡献者贡献明细}
	\label{tab:contrib}
	\begin{tabular}{cp{10cm}}
		\toprule
		姓名   & 贡献                                                                     \\
		\midrule
		葛煦旸 & 主要负责项目主要框架的搭建。
		\begin{itemize}
			\item \verb|dataset| 中数据集与数据加载器的开发;
			\item \verb|core| 中的张量特征抽象;
			\item \verb|nn| 中的 \verb|Trainable| 与 \verb|Module| 的设计,及一些最基本的模型实现;
			\item \verb|raddar_derive| 中的过程宏;
		\end{itemize}                                                                      \\
		许冬   & 主要负责 \verb|raddar_array| 子项目的开发。
		\begin{itemize}
			\item \verb|raddar_array| 中动态计算图、自动微分、动态类型等各个部分的结构设计;
			\item \verb|raddar_array| 中所有张量计算和反向传播算法的接口设计和实现;
			\item \verb|raddar_array| 的设计文档和使用文档的撰写。
		\end{itemize}                                                         \\
		舒文韬 & 主要负责 \verb|optim| 和 \verb|nn| 部分的开发。
		\begin{itemize}
			\item \verb|optim| 中优化器、调度器的接口设计和实现;
			\item \verb|nn| 中基本模块的接口设计和实现;
			\item \verb|nn| 中内置模型的接口设计和实现;
		\end{itemize}                                                         \\
		\bottomrule
	\end{tabular}
\end{table}
\nocite{*}
\bibliography{reference}

\appendix
\part*{Appendix}
\section{Raddar Array 实现概述}\label{app:rad_arr_impl}
\subsection{简介}
\subsection{张量表示}
\subsubsection{去泛型化}
\subsubsection{内存优化}
\subsection{组合视图}
\subsection{结语}
\end{document}